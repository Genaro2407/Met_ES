% Options for packages loaded elsewhere
\PassOptionsToPackage{unicode}{hyperref}
\PassOptionsToPackage{hyphens}{url}
%
\documentclass[
]{article}
\usepackage{amsmath,amssymb}
\usepackage{lmodern}
\usepackage{iftex}
\ifPDFTeX
  \usepackage[T1]{fontenc}
  \usepackage[utf8]{inputenc}
  \usepackage{textcomp} % provide euro and other symbols
\else % if luatex or xetex
  \usepackage{unicode-math}
  \defaultfontfeatures{Scale=MatchLowercase}
  \defaultfontfeatures[\rmfamily]{Ligatures=TeX,Scale=1}
\fi
% Use upquote if available, for straight quotes in verbatim environments
\IfFileExists{upquote.sty}{\usepackage{upquote}}{}
\IfFileExists{microtype.sty}{% use microtype if available
  \usepackage[]{microtype}
  \UseMicrotypeSet[protrusion]{basicmath} % disable protrusion for tt fonts
}{}
\makeatletter
\@ifundefined{KOMAClassName}{% if non-KOMA class
  \IfFileExists{parskip.sty}{%
    \usepackage{parskip}
  }{% else
    \setlength{\parindent}{0pt}
    \setlength{\parskip}{6pt plus 2pt minus 1pt}}
}{% if KOMA class
  \KOMAoptions{parskip=half}}
\makeatother
\usepackage{xcolor}
\usepackage[margin=1in]{geometry}
\usepackage{color}
\usepackage{fancyvrb}
\newcommand{\VerbBar}{|}
\newcommand{\VERB}{\Verb[commandchars=\\\{\}]}
\DefineVerbatimEnvironment{Highlighting}{Verbatim}{commandchars=\\\{\}}
% Add ',fontsize=\small' for more characters per line
\usepackage{framed}
\definecolor{shadecolor}{RGB}{248,248,248}
\newenvironment{Shaded}{\begin{snugshade}}{\end{snugshade}}
\newcommand{\AlertTok}[1]{\textcolor[rgb]{0.94,0.16,0.16}{#1}}
\newcommand{\AnnotationTok}[1]{\textcolor[rgb]{0.56,0.35,0.01}{\textbf{\textit{#1}}}}
\newcommand{\AttributeTok}[1]{\textcolor[rgb]{0.77,0.63,0.00}{#1}}
\newcommand{\BaseNTok}[1]{\textcolor[rgb]{0.00,0.00,0.81}{#1}}
\newcommand{\BuiltInTok}[1]{#1}
\newcommand{\CharTok}[1]{\textcolor[rgb]{0.31,0.60,0.02}{#1}}
\newcommand{\CommentTok}[1]{\textcolor[rgb]{0.56,0.35,0.01}{\textit{#1}}}
\newcommand{\CommentVarTok}[1]{\textcolor[rgb]{0.56,0.35,0.01}{\textbf{\textit{#1}}}}
\newcommand{\ConstantTok}[1]{\textcolor[rgb]{0.00,0.00,0.00}{#1}}
\newcommand{\ControlFlowTok}[1]{\textcolor[rgb]{0.13,0.29,0.53}{\textbf{#1}}}
\newcommand{\DataTypeTok}[1]{\textcolor[rgb]{0.13,0.29,0.53}{#1}}
\newcommand{\DecValTok}[1]{\textcolor[rgb]{0.00,0.00,0.81}{#1}}
\newcommand{\DocumentationTok}[1]{\textcolor[rgb]{0.56,0.35,0.01}{\textbf{\textit{#1}}}}
\newcommand{\ErrorTok}[1]{\textcolor[rgb]{0.64,0.00,0.00}{\textbf{#1}}}
\newcommand{\ExtensionTok}[1]{#1}
\newcommand{\FloatTok}[1]{\textcolor[rgb]{0.00,0.00,0.81}{#1}}
\newcommand{\FunctionTok}[1]{\textcolor[rgb]{0.00,0.00,0.00}{#1}}
\newcommand{\ImportTok}[1]{#1}
\newcommand{\InformationTok}[1]{\textcolor[rgb]{0.56,0.35,0.01}{\textbf{\textit{#1}}}}
\newcommand{\KeywordTok}[1]{\textcolor[rgb]{0.13,0.29,0.53}{\textbf{#1}}}
\newcommand{\NormalTok}[1]{#1}
\newcommand{\OperatorTok}[1]{\textcolor[rgb]{0.81,0.36,0.00}{\textbf{#1}}}
\newcommand{\OtherTok}[1]{\textcolor[rgb]{0.56,0.35,0.01}{#1}}
\newcommand{\PreprocessorTok}[1]{\textcolor[rgb]{0.56,0.35,0.01}{\textit{#1}}}
\newcommand{\RegionMarkerTok}[1]{#1}
\newcommand{\SpecialCharTok}[1]{\textcolor[rgb]{0.00,0.00,0.00}{#1}}
\newcommand{\SpecialStringTok}[1]{\textcolor[rgb]{0.31,0.60,0.02}{#1}}
\newcommand{\StringTok}[1]{\textcolor[rgb]{0.31,0.60,0.02}{#1}}
\newcommand{\VariableTok}[1]{\textcolor[rgb]{0.00,0.00,0.00}{#1}}
\newcommand{\VerbatimStringTok}[1]{\textcolor[rgb]{0.31,0.60,0.02}{#1}}
\newcommand{\WarningTok}[1]{\textcolor[rgb]{0.56,0.35,0.01}{\textbf{\textit{#1}}}}
\usepackage{graphicx}
\makeatletter
\def\maxwidth{\ifdim\Gin@nat@width>\linewidth\linewidth\else\Gin@nat@width\fi}
\def\maxheight{\ifdim\Gin@nat@height>\textheight\textheight\else\Gin@nat@height\fi}
\makeatother
% Scale images if necessary, so that they will not overflow the page
% margins by default, and it is still possible to overwrite the defaults
% using explicit options in \includegraphics[width, height, ...]{}
\setkeys{Gin}{width=\maxwidth,height=\maxheight,keepaspectratio}
% Set default figure placement to htbp
\makeatletter
\def\fps@figure{htbp}
\makeatother
\setlength{\emergencystretch}{3em} % prevent overfull lines
\providecommand{\tightlist}{%
  \setlength{\itemsep}{0pt}\setlength{\parskip}{0pt}}
\setcounter{secnumdepth}{-\maxdimen} % remove section numbering
\ifLuaTeX
  \usepackage{selnolig}  % disable illegal ligatures
\fi
\IfFileExists{bookmark.sty}{\usepackage{bookmark}}{\usepackage{hyperref}}
\IfFileExists{xurl.sty}{\usepackage{xurl}}{} % add URL line breaks if available
\urlstyle{same} % disable monospaced font for URLs
\hypersetup{
  pdftitle={01\_Prueba\_g\_una\_muestra.R},
  pdfauthor={Usuario},
  hidelinks,
  pdfcreator={LaTeX via pandoc}}

\title{01\_Prueba\_g\_una\_muestra.R}
\author{Usuario}
\date{2023-08-22}

\begin{document}
\maketitle

\begin{Shaded}
\begin{Highlighting}[]
\CommentTok{\# Genaro Sánchez Tovar}
\CommentTok{\# 21/Agosto/23}
\CommentTok{\# Matricula: 2133642}


\CommentTok{\# Importar datos {-}{-}{-}{-}{-}{-}{-}{-}{-}{-}{-}{-}{-}{-}{-}{-}{-}{-}{-}{-}{-}{-}{-}{-}{-}{-}{-}{-}{-}{-}{-}{-}{-}{-}{-}{-}{-}{-}{-}{-}{-}{-}{-}{-}{-}{-}{-}{-}{-}{-}{-}{-}{-}{-}{-}{-}{-}{-}}
\CommentTok{\# Funcion read.csv (sirve para importar datos csv a R)}

\FunctionTok{setwd}\NormalTok{(}\StringTok{"C:/Genaro Met.ES/Met\_ES/Scripts"}\NormalTok{)}
\NormalTok{mediciones }\OtherTok{\textless{}{-}} \FunctionTok{read.csv}\NormalTok{(}\StringTok{"mediciones.csv"}\NormalTok{, }\AttributeTok{header =} \ConstantTok{TRUE}\NormalTok{)}
\FunctionTok{head}\NormalTok{(mediciones) }\CommentTok{\# funcion head (sirve para ver los primeros 6 datos)}
\end{Highlighting}
\end{Shaded}

\begin{verbatim}
##   Altura
## 1    8.4
## 2   10.3
## 3   12.4
## 4    9.7
## 5    8.6
## 6    9.3
\end{verbatim}

\begin{Shaded}
\begin{Highlighting}[]
\CommentTok{\# Descriptivas {-}{-}{-}{-}{-}{-}{-}{-}{-}{-}{-}{-}{-}{-}{-}{-}{-}{-}{-}{-}{-}{-}{-}{-}{-}{-}{-}{-}{-}{-}{-}{-}{-}{-}{-}{-}{-}{-}{-}{-}{-}{-}{-}{-}{-}{-}{-}{-}{-}{-}{-}{-}{-}{-}{-}{-}{-}{-}{-}{-}}

\CommentTok{\# medidas de tendencia central: media, mediana, rango...}

\FunctionTok{mean}\NormalTok{(mediciones}\SpecialCharTok{$}\NormalTok{Altura) }\CommentTok{\# Promedio (media)}
\end{Highlighting}
\end{Shaded}

\begin{verbatim}
## [1] 10.17429
\end{verbatim}

\begin{Shaded}
\begin{Highlighting}[]
\FunctionTok{median}\NormalTok{(mediciones}\SpecialCharTok{$}\NormalTok{Altura) }\CommentTok{\# Mediana}
\end{Highlighting}
\end{Shaded}

\begin{verbatim}
## [1] 10.2
\end{verbatim}

\begin{Shaded}
\begin{Highlighting}[]
\FunctionTok{range}\NormalTok{(mediciones}\SpecialCharTok{$}\NormalTok{Altura) }\CommentTok{\# Rango (muestra el primer y ultimo dato)}
\end{Highlighting}
\end{Shaded}

\begin{verbatim}
## [1]  8.1 12.5
\end{verbatim}

\begin{Shaded}
\begin{Highlighting}[]
\FunctionTok{fivenum}\NormalTok{(mediciones}\SpecialCharTok{$}\NormalTok{Altura) }\CommentTok{\# Representa los 5 numeros del boxplot}
\end{Highlighting}
\end{Shaded}

\begin{verbatim}
## [1]  8.10  9.55 10.20 10.75 12.50
\end{verbatim}

\begin{Shaded}
\begin{Highlighting}[]
\CommentTok{\# Medidas de dispersión: Desviacion estandar, varianza...}

\FunctionTok{sd}\NormalTok{(mediciones}\SpecialCharTok{$}\NormalTok{Altura) }\CommentTok{\# Desviacion estandar}
\end{Highlighting}
\end{Shaded}

\begin{verbatim}
## [1] 1.22122
\end{verbatim}

\begin{Shaded}
\begin{Highlighting}[]
\FunctionTok{var}\NormalTok{(mediciones}\SpecialCharTok{$}\NormalTok{Altura) }\CommentTok{\# Varianza}
\end{Highlighting}
\end{Shaded}

\begin{verbatim}
## [1] 1.491378
\end{verbatim}

\begin{Shaded}
\begin{Highlighting}[]
\CommentTok{\# Graficas {-}{-}{-}{-}{-}{-}{-}{-}{-}{-}{-}{-}{-}{-}{-}{-}{-}{-}{-}{-}{-}{-}{-}{-}{-}{-}{-}{-}{-}{-}{-}{-}{-}{-}{-}{-}{-}{-}{-}{-}{-}{-}{-}{-}{-}{-}{-}{-}{-}{-}{-}{-}{-}{-}{-}{-}{-}{-}{-}{-}{-}{-}{-}{-}}

\FunctionTok{boxplot}\NormalTok{(mediciones}\SpecialCharTok{$}\NormalTok{Altura, }\AttributeTok{col =} \StringTok{"lightgreen"}\NormalTok{, }\AttributeTok{ylab =} \StringTok{"Altura 8cm"}\NormalTok{, }\AttributeTok{main =} \StringTok{"Sitio 1"}\NormalTok{, }\AttributeTok{ylim =} \FunctionTok{c}\NormalTok{(}\DecValTok{6}\NormalTok{,}\DecValTok{14}\NormalTok{))}
\end{Highlighting}
\end{Shaded}

\includegraphics{01_Prueba_g_una_muestra_files/figure-latex/unnamed-chunk-1-1.pdf}

\begin{Shaded}
\begin{Highlighting}[]
\CommentTok{\# Hipótesis {-}{-}{-}{-}{-}{-}{-}{-}{-}{-}{-}{-}{-}{-}{-}{-}{-}{-}{-}{-}{-}{-}{-}{-}{-}{-}{-}{-}{-}{-}{-}{-}{-}{-}{-}{-}{-}{-}{-}{-}{-}{-}{-}{-}{-}{-}{-}{-}{-}{-}{-}{-}{-}{-}{-}{-}{-}{-}{-}{-}{-}{-}{-}}

\CommentTok{\# xobs = 10.17 vs xteo = 11}
\CommentTok{\# Los bortes de cedro deberian alcanzar una altura de 11 cm en un año de acuerdo a comentarios de viveristas}
\CommentTok{\# El valor de alta referencia es 0.05}

\CommentTok{\# Procedimiento {-}{-}{-}{-}{-}{-}{-}{-}{-}{-}{-}{-}{-}{-}{-}{-}{-}{-}{-}{-}{-}{-}{-}{-}{-}{-}{-}{-}{-}{-}{-}{-}{-}{-}{-}{-}{-}{-}{-}{-}{-}{-}{-}{-}{-}{-}{-}{-}{-}{-}{-}{-}{-}{-}{-}{-}{-}{-}{-}}

\CommentTok{\# Aplicar la funcion t.test}

\FunctionTok{t.test}\NormalTok{(mediciones}\SpecialCharTok{$}\NormalTok{Altura, }\AttributeTok{mu =} \DecValTok{11}\NormalTok{ )}
\end{Highlighting}
\end{Shaded}

\begin{verbatim}
## 
##  One Sample t-test
## 
## data:  mediciones$Altura
## t = -4.0001, df = 34, p-value = 0.0003237
## alternative hypothesis: true mean is not equal to 11
## 95 percent confidence interval:
##   9.754782 10.593789
## sample estimates:
## mean of x 
##  10.17429
\end{verbatim}

\begin{Shaded}
\begin{Highlighting}[]
\CommentTok{\# "df" = grados de libertad}

\FunctionTok{t.test}\NormalTok{(mediciones}\SpecialCharTok{$}\NormalTok{Altura, }\AttributeTok{mu =} \FloatTok{10.5}\NormalTok{)}
\end{Highlighting}
\end{Shaded}

\begin{verbatim}
## 
##  One Sample t-test
## 
## data:  mediciones$Altura
## t = -1.5779, df = 34, p-value = 0.1239
## alternative hypothesis: true mean is not equal to 10.5
## 95 percent confidence interval:
##   9.754782 10.593789
## sample estimates:
## mean of x 
##  10.17429
\end{verbatim}

\begin{Shaded}
\begin{Highlighting}[]
\FunctionTok{t.test}\NormalTok{(mediciones}\SpecialCharTok{$}\NormalTok{Altura, }\AttributeTok{mu =} \FloatTok{10.6}\NormalTok{)}
\end{Highlighting}
\end{Shaded}

\begin{verbatim}
## 
##  One Sample t-test
## 
## data:  mediciones$Altura
## t = -2.0623, df = 34, p-value = 0.04688
## alternative hypothesis: true mean is not equal to 10.6
## 95 percent confidence interval:
##   9.754782 10.593789
## sample estimates:
## mean of x 
##  10.17429
\end{verbatim}

\begin{Shaded}
\begin{Highlighting}[]
\FunctionTok{t.test}\NormalTok{(mediciones}\SpecialCharTok{$}\NormalTok{Altura, }\AttributeTok{mu =} \FloatTok{10.55}\NormalTok{)}
\end{Highlighting}
\end{Shaded}

\begin{verbatim}
## 
##  One Sample t-test
## 
## data:  mediciones$Altura
## t = -1.8201, df = 34, p-value = 0.07756
## alternative hypothesis: true mean is not equal to 10.55
## 95 percent confidence interval:
##   9.754782 10.593789
## sample estimates:
## mean of x 
##  10.17429
\end{verbatim}

\end{document}
